\chapter{Metodologia}
A metodologia de trabalho prevista compreende uma análise bibliográfica extensiva sobre os requisitos para adoção de um padrão de livre acesso à conteúdo científico. O objetivo desta etapa, além de verificar as vantagens desse modelo, é garantir que autores e revisores tenham seus direitos mantidos, evitando assim possíveis penalidades ou perdas de garantias para ambos. Em seguida, avaliaremos métodos para execução de revisão aberta (\textit{open peer review}), conforme citado em \cite{InteractivePeerReview2010}. Da mesma forma como no estudo anterior, o objetivo é entender os modelos possíveis para implementação de um modelo de revisão aberta e quais as vantagens e desvantagens de cada abordagem.

Uma vez que os conceitos fundamentais tenham sido entendidos, será iniciada a avaliação das plataformas de \textit{blokchain} disponíveis para utilização. Nesta etapa será realizada uma avaliação para verificação das suas funcionalidades. De modo concomitante, será conduzida uma avaliação do suporte à criação de \textit{smart contracts} em cada plataforma. Esse etapa é fundamental uma vez que é necessário garantir que a linguagem a ser utilizada para definição do contrato tenha expressividade suficiente para descrever todas as cláusulas necessárias. Por fim, em virtude dos custos associados à utilização de \textit{blockchain} para armazenamento de dados, avaliaremos opções fora da \textit{chain} para persistência de dados. Nesta etapa avaliaremos ferramentas como IPFS\footnote{https://ipfs.io/}, por exemplo.

A próxima etapa compreende a implementação de um protótipo cujo objetivo é fornecer suporte ao processo de coleta de colaboradores e desenvolvimento do trabalho de maneira colaborativa entre os participantes. Considerando que os envolvidos no desenvolvimento do trabalho tenham finalizado seu esforço inicial, deve ser possível efetuar a submissão para revisão. Nesta fase é importante verificar que os revisores sejam alocados de forma adequada para que o resultado da revisão agregue e garanta mais qualidade ao trabalho final. É importante que o protótipo da plataforma atenta aos requisitos citados na seção anterior deste documento. Esta fase também compreende o suporte ao acesso livre ao conteúdo, bem como a implementação de um modelo de negócios sustentável para garantir a evolução da plataforma, além de recompensar os envolvidos em todo o processo.

Por fim, após as etapas de criação do protótipo e avaliação, será desenvolvida a redação da dissertação e de um artigo científico. O meio de publicação do artigo (conferência ou \textit{journal}) será definido ao longo do desenvolvimento da dissertação.