\keyword{blockchain}
\keyword{plos}

\begin{abstract}
A liberdade para criação e acesso à informação evoluiu drasticamente com o advento da Web. Embora essa evolução possa ser observada em todas as áreas de conhecimento, o acesso ao conteúdo científico ainda está restrito sob o controle de algumas instituições. Embora a elaboração de trabalhos científicos ser muitas vezes fomentada por órgãos do estado, o acesso a esse conteúdo fica restrito a instituições ou indivíduos que se dispõem a pagar por ele. Uma das iniciativas mais influentes que tenta reverter essa situação é o Public Library of Science (PLOS). O PLOS é uma entidade cujo objetivo é tornar livre o acesso às publicações acadêmicas, através de um modelo no qual os autores arcam com os custos de avaliação e possivelmente publicação do trabalho. Este mecanismo por si só não garante a legitimidade das avaliações já que os revisores ainda não são conhecidos, além de não reverter qualquer tipo de financiamento para os autores. Sendo assim, este trabalho tem como objetivo propor uma plataforma aberta para edição, revisão e acesso a trabalhos científicos, utilizando como base para isso a tecnologia de \textit{blockchain}. O emprego de blockchain para este fim traz consigo alguns benefícios inerentes a sua utilização tais como durabilidade, confiabilidade e longevidade, já que os dados inseridos na {\em chain} não podem ser alterados. Outros pontos que serão abordados nesse trabalho são a remuneração dos autores pelo seu trabalho, além do processo de revisão pelos pares.
\end{abstract}
