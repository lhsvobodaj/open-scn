\chapter{Introdução e Motivação}
A quantidade de informação disponível vem crescendo exponencialmente ao longo dos últimos anos. Este crescimento afeta todas as áreas de conhecimento, desde informações livremente publicadas em meios como a Web, até trabalhos acadêmicos os quais são submetidos a processos de revisão antes de se tornarem públicos. Neste contexto, destaca-se o aumento da quantidade de trabalhos científicos. Utilizando como base o banco de dados de publicações da área de computação, o DBLP,\footnote{http://dblp.uni-trier.de/statistics/publicationsperyear.html}, verificamos um crescimento constante de aproximadamente 3\% ao longo dos últimos 5 anos. Em \cite{Online2001} o autor comenta que a quantidade de conteúdo científico de longe excede a capacidade dos pesquisadores de consumí-lo. Além disso, neste mesmo trabalho é ressaltada a importância das publicações livres, ou \textit{open-access publications}. Em \cite{Comparing2004} é feita uma comparação entre publicações abertas em relação ao modelo tradicional. O resultado, identificou-se uma correlação forte entre o número de citações de um trabalho com o fato de eles estar disponível livremente para consulta. Assim, em \cite{Online2001}, o autor recomenda como estratégia para maximizar o impacto e acelerar a evolução da Ciência, que autores e editores facilitem o acesso ao conteúdo científico.

Embora tenhamos vivenciado uma evolução na forma de distribuição do conhecimento científico através dos meios eletrônicos, nem sempre ele está disponível para os leitores por estar sob controle de grandes editores, tornando por vezes o acesso a ele tão caro quanto se fossem distribuídos de maneira física \cite{OpenAccessAnalysis2004}.

Existe um interesse crescente da comunidade acadêmica em publicações livremente distribuídas. Esse formato de publicação compreende a disponibilização do conteúdo da pesquisa de maneira livre e aberta, principalmente através de meios eletrônicos. A busca por publicações abertas se dá principalmente pelo alcance que podem ter e pela velocidade com que o conteúdo está disponível para o público. Além disso, estudos mostram que após determinado período após a publicação da pesquisa de forma aberta (ou mesmo cobrando taxas irrisórias), mais da metade dos pesquisadores estariam dispostos a pagar pelo conteúdo consumido \cite{Societies2002Misc}.

Para evidenciar o impacto de publicações abertas, o estudo conduzido em \cite{OpenAccessImpact2004} verifica que mesmo em áreas de pesquisa como Filosofia, cuja adoção por publicações em meios eletrônicos ocorreu tardiamente em relação a outras áreas, o número de citações de conteúdo distribuído livremente é 45\% maior em relação a trabalhos disponibilizados sob algum tipo de controle ou restrição de acesso. Este mesmo trabalho mostra que em outras áreas como Matemática, esse número é ainda maior, chegando a aproximadamente 90\%. A partir dessa análise, os autores concluem que a hipótese inicialmente levantada em \cite{Online2001} para Computação, se aplica também a outras áreas de pesquisa.

Embora diversas iniciativas de criação de \textit{journals} e repositórios para acesso livre a conteúdo científico tenham sido criados, poucos meios relevantes continuam ativos até a atualidade. Como exemplo, podemos citar o arXiv \footnote{https://arxiv.org/}, PLOS \footnote{https://www.plos.org/} e BioMed \footnote{https://www.biomedcentral.com/}, cada um com suas características para submissão, revisão e disponibilização de conteúdo.

Em \cite{OpenAccessAnalysis2004} o autor faz uma avaliação sobre 6 aspectos relacionadas a publicações abertas e como elas seriam abordadas em três cenários distintos: \textit{open access journals}, repositórios abertos e páginas pessoais dos seus autores.

Outro fator que gera efeito no impacto das publicações é a colaboração entre diferentes pesquisadores. De acordo com \cite{ResearchCollaboration1997}, um colaborador poderia ser definido como alguém que fornece alguma entrada ou contribui com parte da pesquisa. Outra definição seria a de uma pessoa que contribui diretamente para os resultados da pesquisa. Conforme podemos observar, a definição de colaborador não é explícita, mas sim uma convenção do grupo ou área em que os pesquisadores atuam.

Uma das formas de avaliar a colaboração de um trabalho é através dos seus autores \cite{Bibliographical1971}. Embora existam críticas em relação a esse método para avaliação de colaboração, já que em alguns casos autores poderiam ser incluídos apenas por razões sociais e não deveriam receber crédito pelo trabalho \cite{Stealing1993}, verificou que o nível de colaboração influencia o impacto do trabalho, se considerarmos que trabalhos com múltiplos autores são mais citados \cite{Bibliometrics1986}.

Uma das principais motivações para a colaboração é o fato de que a pesquisa moderna demanda conhecimento em diversas áreas. Com frequência, apenas um indivíduo não possuí conhecimento suficiente em todas elas. Assim, a opção de colaboração com outros pesquisadores que já possuem tal conhecimento é uma alternativa. Outro fato interessante da colaboração é a diferença de perspectivas e ideias sobre o trabalho, características que enriquecem as discussões e o entendimento sobre os resultados. Outro aspecto da colaboração é que ela pode ser utilizada para medir a qualidade do pesquisador, grupo de pesquisa ou instituição. Como exemplo deste uso podemos citar a avaliação feita pela CAPES dos programas de pós-graduação brasileiros.

A opção pela publicação de conteúdo científico de maneira aberta traz à tona discussões sobre como garantir os direitos autorais sobre a pesquisa, como proceder com um processo de revisão justo e como os autores podem ser recompensados pelo trabalho. Esses e outros aspectos são discutidos em \cite{OpenAccessAnalysis2004}. Além disso, conforme mencionado anteriormente, com a colaboração entre diversos autores para produção de conteúdo, é necessário um método para determinar o nível de colaboração de cada indivíduo no trabalho.

Este trabalho tem como objetivo a elaboração de uma proposta de plataforma aberta e colaborativa para a elaboração e revisão de trabalhos científicos utilizando como tecnologia base Blockchain \cite{Bitcoin2008}e através da utilização de \textit{smart contracts} \cite{SmartContract2017}. Blockchain basicamente consiste em uma rede de nodos na qual eles não precisa haver confiança entre eles e nem um nodo intermediador para tal função \cite{UntanglingBlockchain2017}. Os nodos mantém um conjunto de estados compartilhado e executam transações que alteram e validam esses estados. Podemos dizer que Blockchain é um tipo de estrutura que armazena o histórico de estados e suas mudanças (transações) no qual todos os nodos do sistema concordam com as transações e ordem em que elas ocorrem. Estas características garantem a esta estrutura integridade, durabilidade e longevidade, já que enquanto a maior parte do poder computacional da rede não estiver sendo utilizada para violá-la, há garantia de que os dados não serão alterados nem perdidos \cite{Bitcoin2008}. Por padrão, todas as blockchains possuem o conceito de \textit{smart contract}. No caso da rede do Bitcoin, esse contrato é exclusivo para troca de moedas entre os participantes da rede. Mais recentemente, surgiram implementações de blockchain que permitem aos seus participantes definirem seus próprios contratos, de forma que eles sejam executados na rede quando uma transação é iniciada. O objetivo com a utilização destas tecnologias é garantir os direitos de autoria de trabalhos científicos baseando-se nas garantias oferecidas pelo blockchain. Além disso, essa plataforma poderá prover suporte a uma modelo de revisão aberto onde os revisores são conhecidos e podem ser inclusive recompensados pelas suas revisões.
